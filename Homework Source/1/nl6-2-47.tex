\item\problemnumber{NL6}{2}{47}{\ }{-}
Assume we are using the simple model for floating-point representation as
given in the text (the representation uses a $14$-bit format, $5$ bits for the
exponent with a bias of $15$, a normalized mantissa of $8$ bits and a single
sign bit for the number):
\begin{list}{\textbf{\alph{enumii}}}{\usecounter{enumii}}
    \item Show how the computer would represent the numbers $100.0$ and $0.25$
    using this floating-point format.
    \item Show how the computer would add the two floating-point numbers in
    part (a) by changing one of the numbers so they are both expressed using
    the same power of $2$.
    \item Show how the computer would represent the sum in part (b) using the
    given floating-point representation. What decimal value for the sum is the
    computer actually storing? Explain.
\end{list}
\vskip12pt
\ifanswers
\textcolor{blue}{
\textbf{Answer:}\\
\begin{list}{\textbf{\alph{enumii}}}{\usecounter{enumii}}
    \item Part a answer goes here
    \item Part b answer goes here
    \item Part c answer goes here
\end{list}
}
\newpage
\fi
