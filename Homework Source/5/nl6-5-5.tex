\item\problemnumber{NL6}{5}{5}{\ }{-}
Consider a $32$-bit hexadecimal number stored in memory as follows:\\
\begin{tabular}{|r|c|c|c|c|}
    \hline
    {\bf Address} & $100$ & $101$ & $102$ & $103$ \\ \hline
    {\bf Value} & $2$A & C$2$ & $08$ & $1$B \\ \hline
\end{tabular}
\begin{list}{\textbf{\alph{enumii})}}{\usecounter{enumii}}
    \item If the machine is big endian and uses $2$s complement representation
        for integers, write the $32$-bit integer stored at address $100$ (you
        may write the number in hex.)
    \item If the machine is big endian and the number is an IEEE
        single-precision floating-point value, is the number positive or
        negative?
    \item If the machine is big endian and the number is an IEEE
        single-precision floating\-point value, determine the decimal equivalent
        of the number stored at address $100$ (you may leave your answer in
        scientific notation form, as a number times a power of $2$).
    \item If the machine is little endian and uses $2$s complement representation
        for integers, write the $32$-bit integer stored at address $100$ (you
        may write the number in hex.)
    \item If the machine is little endian and the number is an IEEE
        single-precision floating-point value, is the number positive or
        negative?
    \item If the machine is little endian and the number is an IEEE
        single-precision floating\-point value, determine the decimal equivalent
        of the number stored at address $100$ (you may leave your answer in
        scientific notation form, as a number times a power of $2$).
\end{list}
\vskip12pt
\ifanswers
\textcolor{blue}{
\textbf{Answer:}\\
}
\newpage
\fi
