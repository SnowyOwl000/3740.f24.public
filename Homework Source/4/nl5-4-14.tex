\item\problemnumber{NL5}{4}{14}{\ }{-}
A digital computer has a memory unit with $32$ bits per word. The instruction
set consists of $110$ different operations. All instructions have an operation
code part (opcode) and two address fields: one for a memory address and one
for a register address. This particular system includes eight general-purpose,
user-addressable registers. Registers may be loaded directly from memory, and
memory may be updated directly from the registers. Direct memory-to-memory data
movement operations are not supported. Each instruction is stored in one word
of memory.
\begin{list}{\textbf{\alph{enumii}}}{\usecounter{enumii}}
    \item How many bits are needed for the opcode?\\[12pt]
    \item How many bits are needed to specify the register?\\[12pt]
    \item How many bits are left for the memory address part of the instruction?\\[12pt]
    \item What is the maximum allowable size for memory?\\[12pt]
    \item What is the largest unsigned binary number that can be accommodated in one word of memory?\\[12pt]
\end{list}
\vskip12pt
\ifanswers
\textcolor{blue}{
\textbf{Answer:}\\
\begin{list}{\textbf{\alph{enumii}}}{\usecounter{enumii}}
\item x
\end{list}
}
\newpage
\fi
